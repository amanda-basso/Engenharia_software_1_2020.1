\section{Manutenções em software estudadas em aula}
Processo de modificar um software ou componente após a entrega, com alguns objetivos tais como:
    \begin{itemize}
        \item \textbf{Manutenção corretiva:} O objetivo deste tipo de manutenção é corrigir falhas de software.
        \item \textbf{Manutenção perfectiva:} A manutenção perfectiva visa melhorar a performance e outros atributos do software, por exemplo a pedido do usuário.
        \item \textbf{Manutenção adaptativa:} Em casos como atualizações de infraestrutura, como hardware, é necessário adaptar o novo ambiente.
        \item \textbf{Manutenção preventiva:} Busca melhorar a manutenibilidade ou confiabilidade futuras. Normalmente as melhorias são baseadas em estimativas.
    \end{itemize}

Alguns textos que corroboram com esta divisão podem ser encontrados em: \url{https://edisciplinas.usp.br/pluginfile.php/325655/mod\_resource/content/1/Aula09\_Manutencao.pdf}, \url{https://www.castsoftware.com/glossary/Four-Types-Of-Software-Maintenance-How-They-Help-Your-Organization-Preventive-Perfective-Adaptive-corrective} e \url{https://ecomputernotes.com/software-engineering/types-of-software-maintenance}.

\section{Manutenções em software estudadas em material extra}
A primeira leitura realizada foi no artigo disponível em \url{https://www.devmedia.com.br/manutencao-de-software-definicoes-e-dificuldades-artigo-revista-sql-magazine-86/20402}.
Nele, a divisão dos tipos de manutenção, embora bem similar à divisão estudada nas aulas da disciplina de Engenharia de Software 1, possui apenas três tipos:
\begin{itemize}
    \item \textbf{Manutenção corretiva:} assim como anteriormente descrito, trata-se da manutenção que busca corrigir falhas de software.
    \item \textbf{Manutenção adaptativa:} novamente, como descrito na divisão adotada em aula, trata-se da manutenção que visa adaptar o software a novo ambiente externo, em situações de atualizações de infraestrutura, por exemplo.
    \item \textbf{Manutenção evolutiva:} esse termo, conquanto não exista na divisão das aulas, refere-se apenas às alterações que cumprem o objetivo de agregar novas funcionalidades ao software - como a \textbf{manutenção perfectiva}.
\end{itemize}
Um outro ponto a ressaltar é o fato de não existir \textbf{manutenção preventiva} na divisão explicada pelo artigo em questão. Um outro artigo que aborda a mesma divisão pode ser encontrado pelo link: \url{https://www.opus-software.com.br/manutencao-de-software-definicao/}.
\newline
\par A segunda leitura extra é parte do material da disciplina de Engenharia de Software da Universidade Federal de Minas Gerais (UFMG) e está disponível em \url{https://homepages.dcc.ufmg.br/~andrehora/teaching/es2/4-manutencao-introducao-es2.pdf}.
\newline
\par O slide 13 exibe que a divisão dos tipos de manutenção de software são iguais àquela estudada em aula. No entanto, é interessante notar, no slide 14, de que maneira as quatro divisões são, na verdade, subdivisões de duas divisões da requisição de modificação: \textbf{correção} e \textbf{melhoria}.
\newline
\par A partir da necessidade de \textbf{correção} do software, é possível chegar até à manutenção corretiva, caso busque corrigir falhas presentes, e também à manutenção preventiva, caso seja necessário melhor preparar o software para um contexto futuro. Por outro lado, caso a requisição de modificação esteja vinculada à \textbf{melhoria}, pode-se tanto chegar à manutenção adaptativa - para quando o software precisa ser adaptado a novo ambiente - quanto à manutenção perfectiva, quando há performance ou outros atributos que podem ser melhorados.
\newline
\par Um terceiro ponto de vista acerca das divisões dos tipos de manutenção está disponível no link \url{https://blog.engeman.com.br/tipos-de-manutencao/}. O artigo é focado para sistemas industriais e trata sobre manutenções em máquinas e indústrias, mas é possível estabelecer um paralelo para a Engenharia de Software. Ele divide as manutenções em três tipos:
\begin{itemize}
    \item \textbf{Manutenção corretiva:} De acordo com o texto, trata-se do tipo de manutenção mais antigo, existente desde antes da Segunda Guerra Mundial, ainda para reparos em indústrias não mecanizadas. A manutenção corretiva, portanto, cumpre (e cumpria) o papel de corrigir falhas - problemas mecânicos, defeitos - de modo a fazer com que o processo de produção volte ao normal. De forma análoga, a manutenção corretiva corrige as falhas de software.
    \newline
    \par O texto ainda divide a manutenção corretiva em dois tipos: \textbf{manutenção corretiva não planejada} - correção de falha aleatórias, que surgiram bruscamente - e \textbf{manutenção corretiva planejada} - correção planejada que é realizada após percepção de perda de performance.

    \item \textbf{Manutenção preventiva:} Tem o objetivo de evitar falhas futuras.
    \item \textbf{Manutenção preditiva:} Tem a finalidade de indicar, através de softwares, as condições de funcionamento de uma máquina. Ou seja, a manutenção preditiva permite o monitoramento do aparelho ou sistema.
\end{itemize}
