\section{Metodologia ágil e Scrum}
O projeto está sendo desenvolvido utilizando a metodologia ágil Scrum. Foi realizada uma reunião, com quatro semanas de antecedência da data de entrega final do projeto, na qual foram planejados as divisões de tarefas entre os membros do grupo, o \textit{Product Backlog} e uma estimativa dos \textit{Sprint Backlogs}.
\newline
\par A divisão de tarefas ficou determinada da seguinte forma:
\begin{itemize}
    \item \textit{Scrum Master}: Amanda. Além do papel de desenvolvedora, também ficou encarregada de puxar as reuniões de \textit{Daily Scrum}.
    \item \textit{Product Owner}: Lucas Reis. Responsável por criar o repositório e os \textit{branches}, além de adicionar os colaboradores e atuar como desenvolvedor.
    \item Desenvolvedores: Jorge, Lucas Martins, Matheus e Pabolo. São responsáveis pelo desenvolvimento prático do projeto, atuando desde a criação das tabelas de testes manuais até a finalização do projeto.
\end{itemize}

O \textit{Backlog}, por sua vez, ficou dividido da seguinte forma:
\begin{itemize}
    \item Primeiro \textit{Sprint} (duração: uma semana)
        \begin{itemize}
            \item Testar a inserção de item bibliográfico - \textit{Book}.
            \item Testar a inserção de item bibliográfio - \textit{Article}.
            \item Testar a importação de itens bibliográficos.
        \end{itemize}
    \item Segundo \textit{Sprint} (duração: duas semanas)
        \begin{itemize}
            \item Manutenção na inserção - validar ano.
            \item Manutenção na inserção - validar Bibtexkey.
            \item Adicionar funcionalidade - importar csv.
        \end{itemize}
    \item Terceiro \textit{Sprint} (duração: uma semana)
        \begin{itemize}
            \item Testar a validação do ano.
            \item Testar a validação do Bibtexkey.
            \item Testar a importação de um arquivo csv.
        \end{itemize}
\end{itemize}

\par A \textit{Sprint Review Meeting} foi realizada uma semana após o início do primeiro \textit{Sprint}. Nessa reunião, foram discutidos o andamento da primeira semana de projeto e os detalhes finais para elaboração do primeiro relatório.
\newline
\par Foi feito um esforço no sentido de manter as \textit{Daily Scrums} diárias mas, em alguns dias, não foi possível devido a conflitos de horários entre as disciplinas realizadas pelos membros do grupo. As dificuldades iniciais do projeto estavam relacionadas a quebrar o projeto em partes menores e factíveis ao longo dos sprints, além da organização necessária para trabalhar seguindo uma metodologia ágil.
